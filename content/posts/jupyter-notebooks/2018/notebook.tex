
% Default to the notebook output style

    


% Inherit from the specified cell style.




    
\documentclass[11pt]{article}

    
    
    \usepackage[T1]{fontenc}
    % Nicer default font (+ math font) than Computer Modern for most use cases
    \usepackage{mathpazo}

    % Basic figure setup, for now with no caption control since it's done
    % automatically by Pandoc (which extracts ![](path) syntax from Markdown).
    \usepackage{graphicx}
    % We will generate all images so they have a width \maxwidth. This means
    % that they will get their normal width if they fit onto the page, but
    % are scaled down if they would overflow the margins.
    \makeatletter
    \def\maxwidth{\ifdim\Gin@nat@width>\linewidth\linewidth
    \else\Gin@nat@width\fi}
    \makeatother
    \let\Oldincludegraphics\includegraphics
    % Set max figure width to be 80% of text width, for now hardcoded.
    \renewcommand{\includegraphics}[1]{\Oldincludegraphics[width=.8\maxwidth]{#1}}
    % Ensure that by default, figures have no caption (until we provide a
    % proper Figure object with a Caption API and a way to capture that
    % in the conversion process - todo).
    \usepackage{caption}
    \DeclareCaptionLabelFormat{nolabel}{}
    \captionsetup{labelformat=nolabel}

    \usepackage{adjustbox} % Used to constrain images to a maximum size 
    \usepackage{xcolor} % Allow colors to be defined
    \usepackage{enumerate} % Needed for markdown enumerations to work
    \usepackage{geometry} % Used to adjust the document margins
    \usepackage{amsmath} % Equations
    \usepackage{amssymb} % Equations
    \usepackage{textcomp} % defines textquotesingle
    % Hack from http://tex.stackexchange.com/a/47451/13684:
    \AtBeginDocument{%
        \def\PYZsq{\textquotesingle}% Upright quotes in Pygmentized code
    }
    \usepackage{upquote} % Upright quotes for verbatim code
    \usepackage{eurosym} % defines \euro
    \usepackage[mathletters]{ucs} % Extended unicode (utf-8) support
    \usepackage[utf8x]{inputenc} % Allow utf-8 characters in the tex document
    \usepackage{fancyvrb} % verbatim replacement that allows latex
    \usepackage{grffile} % extends the file name processing of package graphics 
                         % to support a larger range 
    % The hyperref package gives us a pdf with properly built
    % internal navigation ('pdf bookmarks' for the table of contents,
    % internal cross-reference links, web links for URLs, etc.)
    \usepackage{hyperref}
    \usepackage{longtable} % longtable support required by pandoc >1.10
    \usepackage{booktabs}  % table support for pandoc > 1.12.2
    \usepackage[inline]{enumitem} % IRkernel/repr support (it uses the enumerate* environment)
    \usepackage[normalem]{ulem} % ulem is needed to support strikethroughs (\sout)
                                % normalem makes italics be italics, not underlines
    

    
    
    % Colors for the hyperref package
    \definecolor{urlcolor}{rgb}{0,.145,.698}
    \definecolor{linkcolor}{rgb}{.71,0.21,0.01}
    \definecolor{citecolor}{rgb}{.12,.54,.11}

    % ANSI colors
    \definecolor{ansi-black}{HTML}{3E424D}
    \definecolor{ansi-black-intense}{HTML}{282C36}
    \definecolor{ansi-red}{HTML}{E75C58}
    \definecolor{ansi-red-intense}{HTML}{B22B31}
    \definecolor{ansi-green}{HTML}{00A250}
    \definecolor{ansi-green-intense}{HTML}{007427}
    \definecolor{ansi-yellow}{HTML}{DDB62B}
    \definecolor{ansi-yellow-intense}{HTML}{B27D12}
    \definecolor{ansi-blue}{HTML}{208FFB}
    \definecolor{ansi-blue-intense}{HTML}{0065CA}
    \definecolor{ansi-magenta}{HTML}{D160C4}
    \definecolor{ansi-magenta-intense}{HTML}{A03196}
    \definecolor{ansi-cyan}{HTML}{60C6C8}
    \definecolor{ansi-cyan-intense}{HTML}{258F8F}
    \definecolor{ansi-white}{HTML}{C5C1B4}
    \definecolor{ansi-white-intense}{HTML}{A1A6B2}

    % commands and environments needed by pandoc snippets
    % extracted from the output of `pandoc -s`
    \providecommand{\tightlist}{%
      \setlength{\itemsep}{0pt}\setlength{\parskip}{0pt}}
    \DefineVerbatimEnvironment{Highlighting}{Verbatim}{commandchars=\\\{\}}
    % Add ',fontsize=\small' for more characters per line
    \newenvironment{Shaded}{}{}
    \newcommand{\KeywordTok}[1]{\textcolor[rgb]{0.00,0.44,0.13}{\textbf{{#1}}}}
    \newcommand{\DataTypeTok}[1]{\textcolor[rgb]{0.56,0.13,0.00}{{#1}}}
    \newcommand{\DecValTok}[1]{\textcolor[rgb]{0.25,0.63,0.44}{{#1}}}
    \newcommand{\BaseNTok}[1]{\textcolor[rgb]{0.25,0.63,0.44}{{#1}}}
    \newcommand{\FloatTok}[1]{\textcolor[rgb]{0.25,0.63,0.44}{{#1}}}
    \newcommand{\CharTok}[1]{\textcolor[rgb]{0.25,0.44,0.63}{{#1}}}
    \newcommand{\StringTok}[1]{\textcolor[rgb]{0.25,0.44,0.63}{{#1}}}
    \newcommand{\CommentTok}[1]{\textcolor[rgb]{0.38,0.63,0.69}{\textit{{#1}}}}
    \newcommand{\OtherTok}[1]{\textcolor[rgb]{0.00,0.44,0.13}{{#1}}}
    \newcommand{\AlertTok}[1]{\textcolor[rgb]{1.00,0.00,0.00}{\textbf{{#1}}}}
    \newcommand{\FunctionTok}[1]{\textcolor[rgb]{0.02,0.16,0.49}{{#1}}}
    \newcommand{\RegionMarkerTok}[1]{{#1}}
    \newcommand{\ErrorTok}[1]{\textcolor[rgb]{1.00,0.00,0.00}{\textbf{{#1}}}}
    \newcommand{\NormalTok}[1]{{#1}}
    
    % Additional commands for more recent versions of Pandoc
    \newcommand{\ConstantTok}[1]{\textcolor[rgb]{0.53,0.00,0.00}{{#1}}}
    \newcommand{\SpecialCharTok}[1]{\textcolor[rgb]{0.25,0.44,0.63}{{#1}}}
    \newcommand{\VerbatimStringTok}[1]{\textcolor[rgb]{0.25,0.44,0.63}{{#1}}}
    \newcommand{\SpecialStringTok}[1]{\textcolor[rgb]{0.73,0.40,0.53}{{#1}}}
    \newcommand{\ImportTok}[1]{{#1}}
    \newcommand{\DocumentationTok}[1]{\textcolor[rgb]{0.73,0.13,0.13}{\textit{{#1}}}}
    \newcommand{\AnnotationTok}[1]{\textcolor[rgb]{0.38,0.63,0.69}{\textbf{\textit{{#1}}}}}
    \newcommand{\CommentVarTok}[1]{\textcolor[rgb]{0.38,0.63,0.69}{\textbf{\textit{{#1}}}}}
    \newcommand{\VariableTok}[1]{\textcolor[rgb]{0.10,0.09,0.49}{{#1}}}
    \newcommand{\ControlFlowTok}[1]{\textcolor[rgb]{0.00,0.44,0.13}{\textbf{{#1}}}}
    \newcommand{\OperatorTok}[1]{\textcolor[rgb]{0.40,0.40,0.40}{{#1}}}
    \newcommand{\BuiltInTok}[1]{{#1}}
    \newcommand{\ExtensionTok}[1]{{#1}}
    \newcommand{\PreprocessorTok}[1]{\textcolor[rgb]{0.74,0.48,0.00}{{#1}}}
    \newcommand{\AttributeTok}[1]{\textcolor[rgb]{0.49,0.56,0.16}{{#1}}}
    \newcommand{\InformationTok}[1]{\textcolor[rgb]{0.38,0.63,0.69}{\textbf{\textit{{#1}}}}}
    \newcommand{\WarningTok}[1]{\textcolor[rgb]{0.38,0.63,0.69}{\textbf{\textit{{#1}}}}}
    
    
    % Define a nice break command that doesn't care if a line doesn't already
    % exist.
    \def\br{\hspace*{\fill} \\* }
    % Math Jax compatability definitions
    \def\gt{>}
    \def\lt{<}
    % Document parameters
    \title{Voy\_preso\_o\_no\_analisis\_dataset\_penal2016\_python}
    
    
    

    % Pygments definitions
    
\makeatletter
\def\PY@reset{\let\PY@it=\relax \let\PY@bf=\relax%
    \let\PY@ul=\relax \let\PY@tc=\relax%
    \let\PY@bc=\relax \let\PY@ff=\relax}
\def\PY@tok#1{\csname PY@tok@#1\endcsname}
\def\PY@toks#1+{\ifx\relax#1\empty\else%
    \PY@tok{#1}\expandafter\PY@toks\fi}
\def\PY@do#1{\PY@bc{\PY@tc{\PY@ul{%
    \PY@it{\PY@bf{\PY@ff{#1}}}}}}}
\def\PY#1#2{\PY@reset\PY@toks#1+\relax+\PY@do{#2}}

\expandafter\def\csname PY@tok@w\endcsname{\def\PY@tc##1{\textcolor[rgb]{0.73,0.73,0.73}{##1}}}
\expandafter\def\csname PY@tok@c\endcsname{\let\PY@it=\textit\def\PY@tc##1{\textcolor[rgb]{0.25,0.50,0.50}{##1}}}
\expandafter\def\csname PY@tok@cp\endcsname{\def\PY@tc##1{\textcolor[rgb]{0.74,0.48,0.00}{##1}}}
\expandafter\def\csname PY@tok@k\endcsname{\let\PY@bf=\textbf\def\PY@tc##1{\textcolor[rgb]{0.00,0.50,0.00}{##1}}}
\expandafter\def\csname PY@tok@kp\endcsname{\def\PY@tc##1{\textcolor[rgb]{0.00,0.50,0.00}{##1}}}
\expandafter\def\csname PY@tok@kt\endcsname{\def\PY@tc##1{\textcolor[rgb]{0.69,0.00,0.25}{##1}}}
\expandafter\def\csname PY@tok@o\endcsname{\def\PY@tc##1{\textcolor[rgb]{0.40,0.40,0.40}{##1}}}
\expandafter\def\csname PY@tok@ow\endcsname{\let\PY@bf=\textbf\def\PY@tc##1{\textcolor[rgb]{0.67,0.13,1.00}{##1}}}
\expandafter\def\csname PY@tok@nb\endcsname{\def\PY@tc##1{\textcolor[rgb]{0.00,0.50,0.00}{##1}}}
\expandafter\def\csname PY@tok@nf\endcsname{\def\PY@tc##1{\textcolor[rgb]{0.00,0.00,1.00}{##1}}}
\expandafter\def\csname PY@tok@nc\endcsname{\let\PY@bf=\textbf\def\PY@tc##1{\textcolor[rgb]{0.00,0.00,1.00}{##1}}}
\expandafter\def\csname PY@tok@nn\endcsname{\let\PY@bf=\textbf\def\PY@tc##1{\textcolor[rgb]{0.00,0.00,1.00}{##1}}}
\expandafter\def\csname PY@tok@ne\endcsname{\let\PY@bf=\textbf\def\PY@tc##1{\textcolor[rgb]{0.82,0.25,0.23}{##1}}}
\expandafter\def\csname PY@tok@nv\endcsname{\def\PY@tc##1{\textcolor[rgb]{0.10,0.09,0.49}{##1}}}
\expandafter\def\csname PY@tok@no\endcsname{\def\PY@tc##1{\textcolor[rgb]{0.53,0.00,0.00}{##1}}}
\expandafter\def\csname PY@tok@nl\endcsname{\def\PY@tc##1{\textcolor[rgb]{0.63,0.63,0.00}{##1}}}
\expandafter\def\csname PY@tok@ni\endcsname{\let\PY@bf=\textbf\def\PY@tc##1{\textcolor[rgb]{0.60,0.60,0.60}{##1}}}
\expandafter\def\csname PY@tok@na\endcsname{\def\PY@tc##1{\textcolor[rgb]{0.49,0.56,0.16}{##1}}}
\expandafter\def\csname PY@tok@nt\endcsname{\let\PY@bf=\textbf\def\PY@tc##1{\textcolor[rgb]{0.00,0.50,0.00}{##1}}}
\expandafter\def\csname PY@tok@nd\endcsname{\def\PY@tc##1{\textcolor[rgb]{0.67,0.13,1.00}{##1}}}
\expandafter\def\csname PY@tok@s\endcsname{\def\PY@tc##1{\textcolor[rgb]{0.73,0.13,0.13}{##1}}}
\expandafter\def\csname PY@tok@sd\endcsname{\let\PY@it=\textit\def\PY@tc##1{\textcolor[rgb]{0.73,0.13,0.13}{##1}}}
\expandafter\def\csname PY@tok@si\endcsname{\let\PY@bf=\textbf\def\PY@tc##1{\textcolor[rgb]{0.73,0.40,0.53}{##1}}}
\expandafter\def\csname PY@tok@se\endcsname{\let\PY@bf=\textbf\def\PY@tc##1{\textcolor[rgb]{0.73,0.40,0.13}{##1}}}
\expandafter\def\csname PY@tok@sr\endcsname{\def\PY@tc##1{\textcolor[rgb]{0.73,0.40,0.53}{##1}}}
\expandafter\def\csname PY@tok@ss\endcsname{\def\PY@tc##1{\textcolor[rgb]{0.10,0.09,0.49}{##1}}}
\expandafter\def\csname PY@tok@sx\endcsname{\def\PY@tc##1{\textcolor[rgb]{0.00,0.50,0.00}{##1}}}
\expandafter\def\csname PY@tok@m\endcsname{\def\PY@tc##1{\textcolor[rgb]{0.40,0.40,0.40}{##1}}}
\expandafter\def\csname PY@tok@gh\endcsname{\let\PY@bf=\textbf\def\PY@tc##1{\textcolor[rgb]{0.00,0.00,0.50}{##1}}}
\expandafter\def\csname PY@tok@gu\endcsname{\let\PY@bf=\textbf\def\PY@tc##1{\textcolor[rgb]{0.50,0.00,0.50}{##1}}}
\expandafter\def\csname PY@tok@gd\endcsname{\def\PY@tc##1{\textcolor[rgb]{0.63,0.00,0.00}{##1}}}
\expandafter\def\csname PY@tok@gi\endcsname{\def\PY@tc##1{\textcolor[rgb]{0.00,0.63,0.00}{##1}}}
\expandafter\def\csname PY@tok@gr\endcsname{\def\PY@tc##1{\textcolor[rgb]{1.00,0.00,0.00}{##1}}}
\expandafter\def\csname PY@tok@ge\endcsname{\let\PY@it=\textit}
\expandafter\def\csname PY@tok@gs\endcsname{\let\PY@bf=\textbf}
\expandafter\def\csname PY@tok@gp\endcsname{\let\PY@bf=\textbf\def\PY@tc##1{\textcolor[rgb]{0.00,0.00,0.50}{##1}}}
\expandafter\def\csname PY@tok@go\endcsname{\def\PY@tc##1{\textcolor[rgb]{0.53,0.53,0.53}{##1}}}
\expandafter\def\csname PY@tok@gt\endcsname{\def\PY@tc##1{\textcolor[rgb]{0.00,0.27,0.87}{##1}}}
\expandafter\def\csname PY@tok@err\endcsname{\def\PY@bc##1{\setlength{\fboxsep}{0pt}\fcolorbox[rgb]{1.00,0.00,0.00}{1,1,1}{\strut ##1}}}
\expandafter\def\csname PY@tok@kc\endcsname{\let\PY@bf=\textbf\def\PY@tc##1{\textcolor[rgb]{0.00,0.50,0.00}{##1}}}
\expandafter\def\csname PY@tok@kd\endcsname{\let\PY@bf=\textbf\def\PY@tc##1{\textcolor[rgb]{0.00,0.50,0.00}{##1}}}
\expandafter\def\csname PY@tok@kn\endcsname{\let\PY@bf=\textbf\def\PY@tc##1{\textcolor[rgb]{0.00,0.50,0.00}{##1}}}
\expandafter\def\csname PY@tok@kr\endcsname{\let\PY@bf=\textbf\def\PY@tc##1{\textcolor[rgb]{0.00,0.50,0.00}{##1}}}
\expandafter\def\csname PY@tok@bp\endcsname{\def\PY@tc##1{\textcolor[rgb]{0.00,0.50,0.00}{##1}}}
\expandafter\def\csname PY@tok@fm\endcsname{\def\PY@tc##1{\textcolor[rgb]{0.00,0.00,1.00}{##1}}}
\expandafter\def\csname PY@tok@vc\endcsname{\def\PY@tc##1{\textcolor[rgb]{0.10,0.09,0.49}{##1}}}
\expandafter\def\csname PY@tok@vg\endcsname{\def\PY@tc##1{\textcolor[rgb]{0.10,0.09,0.49}{##1}}}
\expandafter\def\csname PY@tok@vi\endcsname{\def\PY@tc##1{\textcolor[rgb]{0.10,0.09,0.49}{##1}}}
\expandafter\def\csname PY@tok@vm\endcsname{\def\PY@tc##1{\textcolor[rgb]{0.10,0.09,0.49}{##1}}}
\expandafter\def\csname PY@tok@sa\endcsname{\def\PY@tc##1{\textcolor[rgb]{0.73,0.13,0.13}{##1}}}
\expandafter\def\csname PY@tok@sb\endcsname{\def\PY@tc##1{\textcolor[rgb]{0.73,0.13,0.13}{##1}}}
\expandafter\def\csname PY@tok@sc\endcsname{\def\PY@tc##1{\textcolor[rgb]{0.73,0.13,0.13}{##1}}}
\expandafter\def\csname PY@tok@dl\endcsname{\def\PY@tc##1{\textcolor[rgb]{0.73,0.13,0.13}{##1}}}
\expandafter\def\csname PY@tok@s2\endcsname{\def\PY@tc##1{\textcolor[rgb]{0.73,0.13,0.13}{##1}}}
\expandafter\def\csname PY@tok@sh\endcsname{\def\PY@tc##1{\textcolor[rgb]{0.73,0.13,0.13}{##1}}}
\expandafter\def\csname PY@tok@s1\endcsname{\def\PY@tc##1{\textcolor[rgb]{0.73,0.13,0.13}{##1}}}
\expandafter\def\csname PY@tok@mb\endcsname{\def\PY@tc##1{\textcolor[rgb]{0.40,0.40,0.40}{##1}}}
\expandafter\def\csname PY@tok@mf\endcsname{\def\PY@tc##1{\textcolor[rgb]{0.40,0.40,0.40}{##1}}}
\expandafter\def\csname PY@tok@mh\endcsname{\def\PY@tc##1{\textcolor[rgb]{0.40,0.40,0.40}{##1}}}
\expandafter\def\csname PY@tok@mi\endcsname{\def\PY@tc##1{\textcolor[rgb]{0.40,0.40,0.40}{##1}}}
\expandafter\def\csname PY@tok@il\endcsname{\def\PY@tc##1{\textcolor[rgb]{0.40,0.40,0.40}{##1}}}
\expandafter\def\csname PY@tok@mo\endcsname{\def\PY@tc##1{\textcolor[rgb]{0.40,0.40,0.40}{##1}}}
\expandafter\def\csname PY@tok@ch\endcsname{\let\PY@it=\textit\def\PY@tc##1{\textcolor[rgb]{0.25,0.50,0.50}{##1}}}
\expandafter\def\csname PY@tok@cm\endcsname{\let\PY@it=\textit\def\PY@tc##1{\textcolor[rgb]{0.25,0.50,0.50}{##1}}}
\expandafter\def\csname PY@tok@cpf\endcsname{\let\PY@it=\textit\def\PY@tc##1{\textcolor[rgb]{0.25,0.50,0.50}{##1}}}
\expandafter\def\csname PY@tok@c1\endcsname{\let\PY@it=\textit\def\PY@tc##1{\textcolor[rgb]{0.25,0.50,0.50}{##1}}}
\expandafter\def\csname PY@tok@cs\endcsname{\let\PY@it=\textit\def\PY@tc##1{\textcolor[rgb]{0.25,0.50,0.50}{##1}}}

\def\PYZbs{\char`\\}
\def\PYZus{\char`\_}
\def\PYZob{\char`\{}
\def\PYZcb{\char`\}}
\def\PYZca{\char`\^}
\def\PYZam{\char`\&}
\def\PYZlt{\char`\<}
\def\PYZgt{\char`\>}
\def\PYZsh{\char`\#}
\def\PYZpc{\char`\%}
\def\PYZdl{\char`\$}
\def\PYZhy{\char`\-}
\def\PYZsq{\char`\'}
\def\PYZdq{\char`\"}
\def\PYZti{\char`\~}
% for compatibility with earlier versions
\def\PYZat{@}
\def\PYZlb{[}
\def\PYZrb{]}
\makeatother


    % Exact colors from NB
    \definecolor{incolor}{rgb}{0.0, 0.0, 0.5}
    \definecolor{outcolor}{rgb}{0.545, 0.0, 0.0}



    
    % Prevent overflowing lines due to hard-to-break entities
    \sloppy 
    % Setup hyperref package
    \hypersetup{
      breaklinks=true,  % so long urls are correctly broken across lines
      colorlinks=true,
      urlcolor=urlcolor,
      linkcolor=linkcolor,
      citecolor=citecolor,
      }
    % Slightly bigger margins than the latex defaults
    
    \geometry{verbose,tmargin=1in,bmargin=1in,lmargin=1in,rmargin=1in}
    
    

    \begin{document}
    
    
    \maketitle
    
    

    
    \hypertarget{toc_container}{}
Tabla de contenido

\begin{verbatim}
<li><a href="#1-bullet">1. Entendimiento del negocio</a>
    <ul>
        <li><a href="#1.1-bullet">1.1. Contexto</a></li>
        <li><a href="#1.2-bullet">1.2. DataSet</a></li>
        <li><a href="#1.3-bullet">1.3. Objetivo</a></li>
    </ul>
</li>
<li><a href="#2-bullet">2. Entendimiento de los datos</a>
    <ul>
        <li><a href="#2.1-bullet">2.1. Atributos</a></li>
        <li><a href="#2.2-bullet">2.2. Análisis de los atributos</a>
            <ul>
                <li><a href="#2.2.1-bullet">2.2.1. Importancia</a></li>
            </ul>
        </li>
    </ul>
</li>
<li><a href="#3-bullet">3. Preparación de los datos</a>
    <ul>
        <li><a href="#3.1-bullet">3.1. Importación de librerías</a></li>
        <li><a href="#3.2-bullet">3.2. Importación de los datos</a></li>
        <li><a href="#3.3-bullet">3.3. Visualización de los datos</a>
            <ul>
                <li><a href="#3.3.1-bullet">3.3.1. Pandas-profiling</a></li>
            </ul>
        </li>
        <li><a href="#3.4-bullet">3.4. Tratamiento de los datos</a>
            <ul>
                <li><a href="#3.4.1-bullet">3.4.1. Sanitizar los datos</a></li>
                <li><a href="#3.4.3-bullet">3.4.2. Tratamiento de datos faltantes</a></li>
                <li><a href="#3.4.2-bullet">3.4.3. Tratamiento de outliers</a></li>
                <li><a href="#3.4.4-bullet">3.4.4. Correlación de atributos</a></li>
                <li><a href="#3.4.5-bullet">3.4.5. Feature extraction</a></li>
                <li><a href="#3.4.6-bullet">3.4.6. Transformaciones de los datos</a></li>
                <li><a href="#3.4.7-bullet">3.4.7. Dimension reduction</a></li>                    
            </ul>
        </li>
    </ul>
</li>
<li><a href="#4-bullet">4. Modelado</a>
    <ul>
        <li><a href="#4.1-bullet">4.1. Preparación del modelado</a></li>
        <li><a href="#4.2-bullet">4.2. Entrenamiento de los modelos</a>
            <ul>
                <li><a href="#4.2.1-bullet">4.2.1. Linear Regression</a></li>
                <li><a href="#4.2.1-bullet">4.2.2. Linear Regression, L1 Regularization</a></li>
            </ul>
        </li>
        <li><a href="#4.3-bullet">4.3. Comparación de modelos</a></li>
        <li><a href="#4.4-bullet">4.4. Feature selection</a></li>
        <li><a href="#4.5-bullet">4.5. Optimización</a></li>
    </ul>
</li>
<li><a href="#5-bullet">5. Evaluación</a></li>
<li><a href="#6-bullet">6. Puesta en producción</a></li>
\end{verbatim}

    \subsection{\#\# 1. Entendimiento del negocio
}\label{entendimiento-del-negocio}

https://catalogodatos.gub.uy/dataset/penal-montevideo-2016

Uruguay está en un clima político complicado. Con las elecciones del
2019 en la puerta, surge un tema que la opocición enfoca mucho sus armas
para plantearle lucha al gobierno de turno. El tema de la
\textbf{SEGURIDAD}.

Este año se batió un nuevo record en Uruguay. Uno que no brinda orgullo:
\href{https://www.elpais.com.uy/informacion/policiales/record-homicidios-muertes-cuatro-meses.html}{Record
de homicidios en cuatro meses}. El tema es que no solamente batimos el
record en homicidios, sino que también han aumentado las
\href{https://www.elobservador.com.uy/nota/homicidios-aumentaron-66-4-y-rapinas-un-55-8-en-el-primer-semestre-de-2018-20188218280}{rapiñas}.
Es más, el gobierno ya reconoce el
\href{https://www.elobservador.com.uy/nota/se-revelo-la-cifra-que-tanto-preocupa-a-bonomi-4-800-rapinas-mas-en-lo-que-va-de-2018-20186291360}{problema}
y muchos sabemos que esto no es porque se realizan más denuncias gracias
a la tecnología disponible.

¿Pero cual es el problema de fondo? ¿Es un tema económico o cultural?

    \subsubsection{1.1. Contexto }\label{contexto}

El problema (o enfoque de algunos uruguayos) es que las leyes no son
realmente fuertes con respecto a los crímenes. Cuando digo leyes, me
refiero a las condenas por dichos crímenes. ¿Pero realmente es así?

Uruguay tiene un
\href{https://www.elpais.com.uy/informacion/uruguay-record-historico-presos.html}{record
histórico en presos}. O sea, los crímenes están siendo identificados y
los criminales están llendo presos. El tema no está ahí.

¿Estará el tema en lo leve que son las leyes? Esto es lo que se analiza
en este post a partir de los datos de los penales del año 2016.

\textless{}br/

    \subsubsection{1.2. DataSet }\label{dataset}

El conjunto se puede obtener del siguiente link:
\href{https://catalogodatos.gub.uy/dataset/72a5f9be-591d-415d-9428-970336f5e50d/resource/7ae84c9c-d04e-4e16-ba94-5e90da5cdf5e/download/basepenal2016.csv}{Datos}

El dataset contiene información sobre procesos concluidos (por persona)
en los Juzgados Letrados en lo Penal de Montevideo en el año 2016.

Este dataset pertenece al gobierno uruguayo y en conjunto con AGESIC
(Agencia de Gobierno electrónico y Sociedad de la Información y del
Conocimiento) lo han publicado en su página de datos abiertos.

    \subsubsection{1.3. Objetivo }\label{objetivo}

En este post tendremos varios objetivos:

\begin{itemize}
\tightlist
\item
  Analizar el conjunto para ver las relaciones entre las variables y
  poder obtener alguna conclusión del comportamiento de las mismas.
\item
  Realizar un modelo de predicción que permita, dado un nuevo crimen,
  predecir si se irá preso o no.
\end{itemize}

    \subsection{\#\# 2. Entendimiento de los datos
}\label{entendimiento-de-los-datos}

Antes de comenzar con el análisis y modelado de la solución del
problema, debemos entender que explican los atributos. En este punto se
comprenden los atributos y se analiza su importancia para el problema y
la solución.

    \subsubsection{2.1. Atributos }\label{atributos}

Según los
\href{https://catalogodatos.gub.uy/dataset/72a5f9be-591d-415d-9428-970336f5e50d/resource/25af959a-6a27-41c1-a762-6fbba24866a4/download/metadatapenal2016.json}{metadatos}
del dataset, el conjunto cuenta con 29 atributos, incluyendo la variable
objetivo.

Los atributos y su descripción son:

\begin{itemize}
\tightlist
\item
  \texttt{id}: Identificación única del expediente. No se repite
  (Integer) \(\rightarrow\) \emph{Tipo:} {\textbf{NUMERIC} ("-1: Sin
  datos")}
\item
  \texttt{turno}: Es la identificación del Turno del Juzgado Letrado de
  Penal (Integer) \(\rightarrow\) \emph{Tipo:} {\textbf{NUMERIC} ("-1:
  Sin datos")}
\item
  \texttt{fecha}: Fecha de iniciado. Corresponde a la fecha del auto de
  procesamiento (Date) \(\rightarrow\) \emph{Tipo:} {\textbf{POLINOMIAL}
  ("01/01/1900: Sin dato"; "08/11/2016"; "05/08/2016"; ...)}
\item
  \texttt{tipo\_proc}: Tipo de procesamiento (Integer) \(\rightarrow\)
  \emph{Tipo:} {\textbf{POLINOMIAL} ("-1: Sin datos"; "1: Con prisión";
  "2: Sin prisión"; "3: No corresponde")}
\item
  \texttt{fsd}: Fecha de subida al despacho. Es la fecha en la que el
  expediente sube al despacho del juez para dictar la sentencia
  definitiva o interlocutoria que finaliza el proceso (Date)
  \(\rightarrow\) \emph{Tipo:} {\textbf{POLINOMIAL} ("01/01/1900: Sin
  dato"; "08/11/2016"; "05/08/2016"; ...)}
\item
  \texttt{fecha\_sent}: Fecha de la Sentencia. Es la fecha en la que el
  juez dictó la sentencia definitiva o interlocutoria que finalizó el
  proceso (Date) \(\rightarrow\) \emph{Tipo:} {\textbf{POLINOMIAL}
  ("01/01/1900: Sin dato"; "08/11/2016"; "05/08/2016"; ...)}
\item
  \texttt{num\_sent}: Número de la Sentencia. Es el número de la
  Sentencia Definitiva o Interlocutoria que da por concluído el proceso
  (Integer) \(\rightarrow\) \emph{Tipo:} {\textbf{NUMERIC} ("-1: Sin
  datos")}
\item
  \texttt{modo}: Modo de conclusión (Integer) \(\rightarrow\)
  \emph{Tipo:} {\textbf{POLINOMIAL} ("-1: Sin datos"; "1: Sentencia
  Definitiva"; "2: Sentencia Interlocutoria - Amnistía", "3: Sentencia
  Interlocutoria - Clausura definitiva"; "4: Sentencia Interlocutoria -
  Gracia"; "5: Sentencia Interlocutoria - Prescripción de la pena"; "6:
  Sentencia Interlocutoria - Prescripción del delito"; "7: Sentencia
  Interlocutoria - Revocación del auto de procesamiento"; "8: Sentencia
  Interlocutoria - Sobreseimiento"; "9: Sentencia Interlocutoria -
  Fallecimiento antes de la condena"; "10: Sentencia Interlocutoria -
  Art. 59, Ley 14412"; "11: Sentencia Interlocutoria - Remisión")}
\item
  \texttt{sexo}: Sexo del encausado (String) \(\rightarrow\)
  \emph{Tipo:} {\textbf{BINOMIAL} ("-1: Sin dato"; "M"; "F")}
\item
  \texttt{edad}: Edad del encausado (Integer) \(\rightarrow\)
  \emph{Tipo:} {\textbf{NUMERIC} ("-1: Sin datos")}
\item
  \texttt{ant}: Antecedentes judiciales (Integer) \(\rightarrow\)
  \emph{Tipo:} {\textbf{POLINOMIAL} ("-1: Sin datos"; "1: Primario"; "2:
  Reincidente"; "3: No corresponde")}
\item
  \texttt{sit\_esp}: Situación especial (Integer) \(\rightarrow\)
  \emph{Tipo:} {\textbf{POLINOMIAL} ("-1: Sin datos"; "1: Cúmulo o
  unificación de sentencias"; "2: Ley de prensa u otros procesos por
  audiencia"; "3: Extradición (se refiere al expediente que tramita la
  extradición de un sujeto y no al exhorto o solicitud administrativa de
  extradición)"; "4: Autor ininputable mayor de edad", "5: Acción de
  amparo"; "6: No corresponde (cuando no corresponde a ninguno de los
  casos que fueron codificados anteriormente)")}
\item
  \texttt{tipo\_del}: Tipo de delito (Integer) \(\rightarrow\)
  \emph{Tipo:} {\textbf{POLINOMIAL} ("-1: Sin dato"; "1: Hurto"; "2:
  Rapiña"; "3: Delitos previstos en la Ley de estupefacientes", "4:
  Receptación"; "5: Lesiones personales"; "6: Homicidio", "7: Homicidio
  culpable"; "8: Lesiones graves"; "9: Estafa"; "10: Violencia privada";
  "11: Apropiación indebida"; "12: Atentado violento al pudor"; "13:
  Desacato"; "14: Atentado"; "15: Ley de cheques"; "16: Falsificación o
  alteración de certificados"; "17: Violencia domóstica"; "18: Uso de
  documento falso público o privado"; "19: Otros"; "20: No
  corresponde")}
\item
  \texttt{tipo\_del2}: Tipo de delito (Integer) \(\rightarrow\)
  \emph{Tipo:} {\textbf{POLINOMIAL} ("-1: Sin dato"; "1: Hurto"; "2:
  Rapiña"; "3: Delitos previstos en la Ley de estupefacientes", "4:
  Receptación"; "5: Lesiones personales"; "6: Homicidio", "7: Homicidio
  culpable"; "8: Lesiones graves"; "9: Estafa"; "10: Violencia privada";
  "11: Apropiación indebida"; "12: Atentado violento al pudor"; "13:
  Desacato"; "14: Atentado"; "15: Ley de cheques"; "16: Falsificación o
  alteración de certificados"; "17: Violencia domóstica"; "18: Uso de
  documento falso público o privado"; "19: Otros"; "20: No
  corresponde")}
\item
  \texttt{tipo\_del3}: Tipo de delito (Integer) \(\rightarrow\)
  \emph{Tipo:} {\textbf{POLINOMIAL} ("-1: Sin dato"; "1: Hurto"; "2:
  Rapiña"; "3: Delitos previstos en la Ley de estupefacientes", "4:
  Receptación"; "5: Lesiones personales"; "6: Homicidio", "7: Homicidio
  culpable"; "8: Lesiones graves"; "9: Estafa"; "10: Violencia privada";
  "11: Apropiación indebida"; "12: Atentado violento al pudor"; "13:
  Desacato"; "14: Atentado"; "15: Ley de cheques"; "16: Falsificación o
  alteración de certificados"; "17: Violencia domóstica"; "18: Uso de
  documento falso público o privado"; "19: Otros"; "20: No
  corresponde")}
\item
  \texttt{otros\_del}: Otro tipo de delito cometido (String)
  \(\rightarrow\) \emph{Tipo:} {\textbf{POLINOMIAL} ("Sin dato";
  "Texto")}
\item
  \texttt{otros2}: Otro tipo de delito cometido (String) \(\rightarrow\)
  \emph{Tipo:} {\textbf{POLINOMIAL} ("Sin dato"; "Texto")}
\item
  \texttt{otros\_del3}: Otro tipo de delito cometido (String)
  \(\rightarrow\) \emph{Tipo:} {\textbf{POLINOMIAL} ("Sin dato";
  "Texto")}
\item
  \texttt{Tent}: Tentativa (Integer) \(\rightarrow\) \emph{Tipo:}
  {\textbf{BINOMINAL} ("-1: Sin dato"; "0: No"; "1: Si")}
\item
  \texttt{Tent\_2}: Tentativa (Integer) \(\rightarrow\) \emph{Tipo:}
  {\textbf{BINOMINAL} ("-1: Sin dato"; "0: No"; "1: Si")}
\item
  \texttt{Tent\_3}: Tentativa (Integer) \(\rightarrow\) \emph{Tipo:}
  {\textbf{BINOMINAL} ("-1: Sin dato"; "0: No"; "1: Si")}
\item
  \texttt{agrav}: Circunstancias agravantes (Integer) \(\rightarrow\)
  \emph{Tipo:} {\textbf{POLINOMIAL} ("-1: Sin dato"; "1: Agravado"; "2:
  Especialmente agravado"; "3: Muy especialmente agravado")}
\item
  \texttt{agrav2}: Circunstancias agravantes (Integer) \(\rightarrow\)
  \emph{Tipo:} {\textbf{POLINOMIAL} ("-1: Sin dato"; "1: Agravado"; "2:
  Especialmente agravado"; "3: Muy especialmente agravado")}
\item
  \texttt{agrav3}: Circunstancias agravantes (Integer) \(\rightarrow\)
  \emph{Tipo:} {\textbf{POLINOMIAL} ("-1: Sin dato"; "1: Agravado"; "2:
  Especialmente agravado"; "3: Muy especialmente agravado")}
\item
  \texttt{multa}: Multa (Integer) \(\rightarrow\) \emph{Tipo:}
  {\textbf{NUMERIC} ("Sin dato"; "Monto")}
\item
  \texttt{monto\_a}: Sentencia en años (Integer) \(\rightarrow\)
  \emph{Tipo:} {\textbf{POLINOMIAL} ("-1: Sin dato"; "Cantidad de
  años")}
\item
  \texttt{monto\_m}: Sentencia en meses (Integer) \(\rightarrow\)
  \emph{Tipo:} {\textbf{POLINOMIAL} ("-1: Sin dato"; "Cantidad de
  meses")}
\item
  \texttt{monto\_d}: Sentencia en dias (Integer) \(\rightarrow\)
  \emph{Tipo:} {\textbf{POLINOMIAL} ("-1: Sin dato"; "Cantidad de
  dias")}
\item
  \texttt{inf\_adicional}: Información adicional (String)
  \(\rightarrow\) \emph{Tipo:} {\textbf{POLINOMIAL} ("Sin dato";
  "Texto")}
\end{itemize}

    \subsubsection{2.2. Análisis de los atributos
}\label{anuxe1lisis-de-los-atributos}

En esta sección se hace un análisis previo de los atributos.

    \paragraph{2.2.1. Importancia }\label{importancia}

La importancia de los atributos con respecto al contexto es muy
importante, ya que atributos que no tienen importancia dado el contexto,
pueden afectar la predicción final.

\begin{itemize}
\tightlist
\item
  Importancia de cada atributo o indicar los que no son importantes.
\end{itemize}

    \subsection{3. Preparación de los datos
}\label{preparaciuxf3n-de-los-datos}

En esta sección se preparan los datos para ser utilizados por varios
modelos. Se tiene en cuenta las restricciones de los modelos, por lo que
se pueden preparar varios conjuntos para distintos modelos.

\begin{itemize}
\tightlist
\item
  Como se enfocan estos problemas usualmente.
\item
  Tipos de algoritmos a utilizar.
\item
  Restricciones de los datos para dichos algoritmos.
\end{itemize}

Ej, para la resgresión lineal: Requerimientos de los atributos para la
regresión lineal:

\begin{enumerate}
\def\labelenumi{\arabic{enumi}.}
\tightlist
\item
  Asumir una relación lineal entre los datos.
\item
  Remover el ruido.
\item
  Remover atributos correlacionados.
\item
  Distribución gaussiana.
\item
  Normalizar entradas.
\end{enumerate}

    \subsubsection{3.1. Importación de librerías
}\label{importaciuxf3n-de-libreruxedas}

Las librerías a utilizar son las siguientes:

\begin{itemize}
\tightlist
\item
  \texttt{numpy} \(\rightarrow\) NumPy es el paquete fundamental para la
  informática científica con Python (http://www.numpy.org/).
\item
  \texttt{pandas} \(\rightarrow\) Pandas provee herramientas para la
  importación y el fácil análisis de los datos
  (https://pandas.pydata.org/).
\item
  \texttt{matplotlib} \(\rightarrow\) Matplotlib permite graficar los
  datos (https://matplotlib.org/).
\item
  \texttt{seaborn} \(\rightarrow\) Seaborn permite una linda
  visualización estadística de los datos, en conjunción con matplotlib
  (https://seaborn.pydata.org/).
\item
  \texttt{pandas\_profiling} \(\rightarrow\) Brinda un completo resumen
  de los datos (https://github.com/pandas-profiling/pandas-profiling).
\end{itemize}

    \begin{Verbatim}[commandchars=\\\{\}]
{\color{incolor}In [{\color{incolor}20}]:} \PY{k+kn}{import} \PY{n+nn}{numpy} \PY{k}{as} \PY{n+nn}{np} 
         \PY{k+kn}{import} \PY{n+nn}{pandas} \PY{k}{as} \PY{n+nn}{pd} 
         
         \PY{k+kn}{import} \PY{n+nn}{seaborn} \PY{k}{as} \PY{n+nn}{sns}
         \PY{k+kn}{from} \PY{n+nn}{matplotlib} \PY{k}{import} \PY{n}{pyplot} \PY{k}{as} \PY{n}{plt}
         \PY{o}{\PYZpc{}}\PY{k}{matplotlib} inline
         \PY{n}{sns}\PY{o}{.}\PY{n}{set\PYZus{}style}\PY{p}{(}\PY{l+s+s2}{\PYZdq{}}\PY{l+s+s2}{whitegrid}\PY{l+s+s2}{\PYZdq{}}\PY{p}{)}
         
         \PY{k+kn}{import} \PY{n+nn}{pandas\PYZus{}profiling}
         
         \PY{k+kn}{import} \PY{n+nn}{warnings}
         \PY{n}{warnings}\PY{o}{.}\PY{n}{filterwarnings}\PY{p}{(}\PY{l+s+s1}{\PYZsq{}}\PY{l+s+s1}{ignore}\PY{l+s+s1}{\PYZsq{}}\PY{p}{)}
\end{Verbatim}


    \begin{Verbatim}[commandchars=\\\{\}]
c:\textbackslash{}programdata\textbackslash{}anaconda3\textbackslash{}envs\textbackslash{}ml-porfolio\textbackslash{}lib\textbackslash{}site-packages\textbackslash{}pandas\_profiling\textbackslash{}plot.py:15: UserWarning: 
This call to matplotlib.use() has no effect because the backend has already
been chosen; matplotlib.use() must be called *before* pylab, matplotlib.pyplot,
or matplotlib.backends is imported for the first time.

The backend was *originally* set to 'module://ipykernel.pylab.backend\_inline' by the following code:
  File "c:\textbackslash{}programdata\textbackslash{}anaconda3\textbackslash{}envs\textbackslash{}ml-porfolio\textbackslash{}lib\textbackslash{}runpy.py", line 193, in \_run\_module\_as\_main
    "\_\_main\_\_", mod\_spec)
  File "c:\textbackslash{}programdata\textbackslash{}anaconda3\textbackslash{}envs\textbackslash{}ml-porfolio\textbackslash{}lib\textbackslash{}runpy.py", line 85, in \_run\_code
    exec(code, run\_globals)
  File "c:\textbackslash{}programdata\textbackslash{}anaconda3\textbackslash{}envs\textbackslash{}ml-porfolio\textbackslash{}lib\textbackslash{}site-packages\textbackslash{}ipykernel\_launcher.py", line 16, in <module>
    app.launch\_new\_instance()
  File "c:\textbackslash{}programdata\textbackslash{}anaconda3\textbackslash{}envs\textbackslash{}ml-porfolio\textbackslash{}lib\textbackslash{}site-packages\textbackslash{}traitlets\textbackslash{}config\textbackslash{}application.py", line 658, in launch\_instance
    app.start()
  File "c:\textbackslash{}programdata\textbackslash{}anaconda3\textbackslash{}envs\textbackslash{}ml-porfolio\textbackslash{}lib\textbackslash{}site-packages\textbackslash{}ipykernel\textbackslash{}kernelapp.py", line 486, in start
    self.io\_loop.start()
  File "c:\textbackslash{}programdata\textbackslash{}anaconda3\textbackslash{}envs\textbackslash{}ml-porfolio\textbackslash{}lib\textbackslash{}site-packages\textbackslash{}tornado\textbackslash{}platform\textbackslash{}asyncio.py", line 132, in start
    self.asyncio\_loop.run\_forever()
  File "c:\textbackslash{}programdata\textbackslash{}anaconda3\textbackslash{}envs\textbackslash{}ml-porfolio\textbackslash{}lib\textbackslash{}asyncio\textbackslash{}base\_events.py", line 422, in run\_forever
    self.\_run\_once()
  File "c:\textbackslash{}programdata\textbackslash{}anaconda3\textbackslash{}envs\textbackslash{}ml-porfolio\textbackslash{}lib\textbackslash{}asyncio\textbackslash{}base\_events.py", line 1434, in \_run\_once
    handle.\_run()
  File "c:\textbackslash{}programdata\textbackslash{}anaconda3\textbackslash{}envs\textbackslash{}ml-porfolio\textbackslash{}lib\textbackslash{}asyncio\textbackslash{}events.py", line 145, in \_run
    self.\_callback(*self.\_args)
  File "c:\textbackslash{}programdata\textbackslash{}anaconda3\textbackslash{}envs\textbackslash{}ml-porfolio\textbackslash{}lib\textbackslash{}site-packages\textbackslash{}tornado\textbackslash{}platform\textbackslash{}asyncio.py", line 122, in \_handle\_events
    handler\_func(fileobj, events)
  File "c:\textbackslash{}programdata\textbackslash{}anaconda3\textbackslash{}envs\textbackslash{}ml-porfolio\textbackslash{}lib\textbackslash{}site-packages\textbackslash{}tornado\textbackslash{}stack\_context.py", line 300, in null\_wrapper
    return fn(*args, **kwargs)
  File "c:\textbackslash{}programdata\textbackslash{}anaconda3\textbackslash{}envs\textbackslash{}ml-porfolio\textbackslash{}lib\textbackslash{}site-packages\textbackslash{}zmq\textbackslash{}eventloop\textbackslash{}zmqstream.py", line 450, in \_handle\_events
    self.\_handle\_recv()
  File "c:\textbackslash{}programdata\textbackslash{}anaconda3\textbackslash{}envs\textbackslash{}ml-porfolio\textbackslash{}lib\textbackslash{}site-packages\textbackslash{}zmq\textbackslash{}eventloop\textbackslash{}zmqstream.py", line 480, in \_handle\_recv
    self.\_run\_callback(callback, msg)
  File "c:\textbackslash{}programdata\textbackslash{}anaconda3\textbackslash{}envs\textbackslash{}ml-porfolio\textbackslash{}lib\textbackslash{}site-packages\textbackslash{}zmq\textbackslash{}eventloop\textbackslash{}zmqstream.py", line 432, in \_run\_callback
    callback(*args, **kwargs)
  File "c:\textbackslash{}programdata\textbackslash{}anaconda3\textbackslash{}envs\textbackslash{}ml-porfolio\textbackslash{}lib\textbackslash{}site-packages\textbackslash{}tornado\textbackslash{}stack\_context.py", line 300, in null\_wrapper
    return fn(*args, **kwargs)
  File "c:\textbackslash{}programdata\textbackslash{}anaconda3\textbackslash{}envs\textbackslash{}ml-porfolio\textbackslash{}lib\textbackslash{}site-packages\textbackslash{}ipykernel\textbackslash{}kernelbase.py", line 283, in dispatcher
    return self.dispatch\_shell(stream, msg)
  File "c:\textbackslash{}programdata\textbackslash{}anaconda3\textbackslash{}envs\textbackslash{}ml-porfolio\textbackslash{}lib\textbackslash{}site-packages\textbackslash{}ipykernel\textbackslash{}kernelbase.py", line 233, in dispatch\_shell
    handler(stream, idents, msg)
  File "c:\textbackslash{}programdata\textbackslash{}anaconda3\textbackslash{}envs\textbackslash{}ml-porfolio\textbackslash{}lib\textbackslash{}site-packages\textbackslash{}ipykernel\textbackslash{}kernelbase.py", line 399, in execute\_request
    user\_expressions, allow\_stdin)
  File "c:\textbackslash{}programdata\textbackslash{}anaconda3\textbackslash{}envs\textbackslash{}ml-porfolio\textbackslash{}lib\textbackslash{}site-packages\textbackslash{}ipykernel\textbackslash{}ipkernel.py", line 208, in do\_execute
    res = shell.run\_cell(code, store\_history=store\_history, silent=silent)
  File "c:\textbackslash{}programdata\textbackslash{}anaconda3\textbackslash{}envs\textbackslash{}ml-porfolio\textbackslash{}lib\textbackslash{}site-packages\textbackslash{}ipykernel\textbackslash{}zmqshell.py", line 537, in run\_cell
    return super(ZMQInteractiveShell, self).run\_cell(*args, **kwargs)
  File "c:\textbackslash{}programdata\textbackslash{}anaconda3\textbackslash{}envs\textbackslash{}ml-porfolio\textbackslash{}lib\textbackslash{}site-packages\textbackslash{}IPython\textbackslash{}core\textbackslash{}interactiveshell.py", line 2662, in run\_cell
    raw\_cell, store\_history, silent, shell\_futures)
  File "c:\textbackslash{}programdata\textbackslash{}anaconda3\textbackslash{}envs\textbackslash{}ml-porfolio\textbackslash{}lib\textbackslash{}site-packages\textbackslash{}IPython\textbackslash{}core\textbackslash{}interactiveshell.py", line 2785, in \_run\_cell
    interactivity=interactivity, compiler=compiler, result=result)
  File "c:\textbackslash{}programdata\textbackslash{}anaconda3\textbackslash{}envs\textbackslash{}ml-porfolio\textbackslash{}lib\textbackslash{}site-packages\textbackslash{}IPython\textbackslash{}core\textbackslash{}interactiveshell.py", line 2901, in run\_ast\_nodes
    if self.run\_code(code, result):
  File "c:\textbackslash{}programdata\textbackslash{}anaconda3\textbackslash{}envs\textbackslash{}ml-porfolio\textbackslash{}lib\textbackslash{}site-packages\textbackslash{}IPython\textbackslash{}core\textbackslash{}interactiveshell.py", line 2961, in run\_code
    exec(code\_obj, self.user\_global\_ns, self.user\_ns)
  File "<ipython-input-20-b994aea22841>", line 6, in <module>
    get\_ipython().run\_line\_magic('matplotlib', 'inline')
  File "c:\textbackslash{}programdata\textbackslash{}anaconda3\textbackslash{}envs\textbackslash{}ml-porfolio\textbackslash{}lib\textbackslash{}site-packages\textbackslash{}IPython\textbackslash{}core\textbackslash{}interactiveshell.py", line 2131, in run\_line\_magic
    result = fn(*args,**kwargs)
  File "<decorator-gen-108>", line 2, in matplotlib
  File "c:\textbackslash{}programdata\textbackslash{}anaconda3\textbackslash{}envs\textbackslash{}ml-porfolio\textbackslash{}lib\textbackslash{}site-packages\textbackslash{}IPython\textbackslash{}core\textbackslash{}magic.py", line 187, in <lambda>
    call = lambda f, *a, **k: f(*a, **k)
  File "c:\textbackslash{}programdata\textbackslash{}anaconda3\textbackslash{}envs\textbackslash{}ml-porfolio\textbackslash{}lib\textbackslash{}site-packages\textbackslash{}IPython\textbackslash{}core\textbackslash{}magics\textbackslash{}pylab.py", line 99, in matplotlib
    gui, backend = self.shell.enable\_matplotlib(args.gui)
  File "c:\textbackslash{}programdata\textbackslash{}anaconda3\textbackslash{}envs\textbackslash{}ml-porfolio\textbackslash{}lib\textbackslash{}site-packages\textbackslash{}IPython\textbackslash{}core\textbackslash{}interactiveshell.py", line 3049, in enable\_matplotlib
    pt.activate\_matplotlib(backend)
  File "c:\textbackslash{}programdata\textbackslash{}anaconda3\textbackslash{}envs\textbackslash{}ml-porfolio\textbackslash{}lib\textbackslash{}site-packages\textbackslash{}IPython\textbackslash{}core\textbackslash{}pylabtools.py", line 311, in activate\_matplotlib
    matplotlib.pyplot.switch\_backend(backend)
  File "c:\textbackslash{}programdata\textbackslash{}anaconda3\textbackslash{}envs\textbackslash{}ml-porfolio\textbackslash{}lib\textbackslash{}site-packages\textbackslash{}matplotlib\textbackslash{}pyplot.py", line 231, in switch\_backend
    matplotlib.use(newbackend, warn=False, force=True)
  File "c:\textbackslash{}programdata\textbackslash{}anaconda3\textbackslash{}envs\textbackslash{}ml-porfolio\textbackslash{}lib\textbackslash{}site-packages\textbackslash{}matplotlib\textbackslash{}\_\_init\_\_.py", line 1422, in use
    reload(sys.modules['matplotlib.backends'])
  File "c:\textbackslash{}programdata\textbackslash{}anaconda3\textbackslash{}envs\textbackslash{}ml-porfolio\textbackslash{}lib\textbackslash{}importlib\textbackslash{}\_\_init\_\_.py", line 166, in reload
    \_bootstrap.\_exec(spec, module)
  File "c:\textbackslash{}programdata\textbackslash{}anaconda3\textbackslash{}envs\textbackslash{}ml-porfolio\textbackslash{}lib\textbackslash{}site-packages\textbackslash{}matplotlib\textbackslash{}backends\textbackslash{}\_\_init\_\_.py", line 16, in <module>
    line for line in traceback.format\_stack()


  matplotlib.use(BACKEND)

    \end{Verbatim}

    \subsubsection{3.2. Importación de los datos
}\label{importaciuxf3n-de-los-datos}

Para importar los datos utilizamos Pandas (\texttt{pd.read\_csv()}), con
los cual los cargamos en la estructura de datos \textbf{Dataframe} de
Pandas. En este punto también se realiza los joins necesarios para
obtener un conjunto con el que trabajar.

    \begin{Verbatim}[commandchars=\\\{\}]
{\color{incolor}In [{\color{incolor}21}]:} \PY{c+c1}{\PYZsh{} Datos de entrenamiento.}
         \PY{n}{training} \PY{o}{=} \PY{n}{pd}\PY{o}{.}\PY{n}{read\PYZus{}csv}\PY{p}{(}\PY{l+s+s2}{\PYZdq{}}\PY{l+s+s2}{datasettraining.csv}\PY{l+s+s2}{\PYZdq{}}\PY{p}{)}
         
         \PY{c+c1}{\PYZsh{} Datos para testing.}
         \PY{n}{testing} \PY{o}{=} \PY{n}{pd}\PY{o}{.}\PY{n}{read\PYZus{}csv}\PY{p}{(}\PY{l+s+s2}{\PYZdq{}}\PY{l+s+s2}{datasettesting.csv}\PY{l+s+s2}{\PYZdq{}}\PY{p}{)}
\end{Verbatim}


    \begin{Verbatim}[commandchars=\\\{\}]

        ---------------------------------------------------------------------------

        FileNotFoundError                         Traceback (most recent call last)

        <ipython-input-21-7f790f5735e2> in <module>()
          1 \# Datos de entrenamiento.
    ----> 2 training = pd.read\_csv("datasettraining.csv")
          3 
          4 \# Datos para testing.
          5 testing = pd.read\_csv("datasettesting.csv")
    

        c:\textbackslash{}programdata\textbackslash{}anaconda3\textbackslash{}envs\textbackslash{}ml-porfolio\textbackslash{}lib\textbackslash{}site-packages\textbackslash{}pandas\textbackslash{}io\textbackslash{}parsers.py in parser\_f(filepath\_or\_buffer, sep, delimiter, header, names, index\_col, usecols, squeeze, prefix, mangle\_dupe\_cols, dtype, engine, converters, true\_values, false\_values, skipinitialspace, skiprows, nrows, na\_values, keep\_default\_na, na\_filter, verbose, skip\_blank\_lines, parse\_dates, infer\_datetime\_format, keep\_date\_col, date\_parser, dayfirst, iterator, chunksize, compression, thousands, decimal, lineterminator, quotechar, quoting, escapechar, comment, encoding, dialect, tupleize\_cols, error\_bad\_lines, warn\_bad\_lines, skipfooter, doublequote, delim\_whitespace, low\_memory, memory\_map, float\_precision)
        676                     skip\_blank\_lines=skip\_blank\_lines)
        677 
    --> 678         return \_read(filepath\_or\_buffer, kwds)
        679 
        680     parser\_f.\_\_name\_\_ = name
    

        c:\textbackslash{}programdata\textbackslash{}anaconda3\textbackslash{}envs\textbackslash{}ml-porfolio\textbackslash{}lib\textbackslash{}site-packages\textbackslash{}pandas\textbackslash{}io\textbackslash{}parsers.py in \_read(filepath\_or\_buffer, kwds)
        438 
        439     \# Create the parser.
    --> 440     parser = TextFileReader(filepath\_or\_buffer, **kwds)
        441 
        442     if chunksize or iterator:
    

        c:\textbackslash{}programdata\textbackslash{}anaconda3\textbackslash{}envs\textbackslash{}ml-porfolio\textbackslash{}lib\textbackslash{}site-packages\textbackslash{}pandas\textbackslash{}io\textbackslash{}parsers.py in \_\_init\_\_(self, f, engine, **kwds)
        785             self.options['has\_index\_names'] = kwds['has\_index\_names']
        786 
    --> 787         self.\_make\_engine(self.engine)
        788 
        789     def close(self):
    

        c:\textbackslash{}programdata\textbackslash{}anaconda3\textbackslash{}envs\textbackslash{}ml-porfolio\textbackslash{}lib\textbackslash{}site-packages\textbackslash{}pandas\textbackslash{}io\textbackslash{}parsers.py in \_make\_engine(self, engine)
       1012     def \_make\_engine(self, engine='c'):
       1013         if engine == 'c':
    -> 1014             self.\_engine = CParserWrapper(self.f, **self.options)
       1015         else:
       1016             if engine == 'python':
    

        c:\textbackslash{}programdata\textbackslash{}anaconda3\textbackslash{}envs\textbackslash{}ml-porfolio\textbackslash{}lib\textbackslash{}site-packages\textbackslash{}pandas\textbackslash{}io\textbackslash{}parsers.py in \_\_init\_\_(self, src, **kwds)
       1706         kwds['usecols'] = self.usecols
       1707 
    -> 1708         self.\_reader = parsers.TextReader(src, **kwds)
       1709 
       1710         passed\_names = self.names is None
    

        pandas\textbackslash{}\_libs\textbackslash{}parsers.pyx in pandas.\_libs.parsers.TextReader.\_\_cinit\_\_()
    

        pandas\textbackslash{}\_libs\textbackslash{}parsers.pyx in pandas.\_libs.parsers.TextReader.\_setup\_parser\_source()
    

        FileNotFoundError: File b'datasettraining.csv' does not exist

    \end{Verbatim}

    \begin{Verbatim}[commandchars=\\\{\}]
{\color{incolor}In [{\color{incolor} }]:} \PY{c+c1}{\PYZsh{} Obtenemos el Id de los datos.}
        \PY{n}{testID} \PY{o}{=} \PY{n}{testing}\PY{p}{[}\PY{l+s+s1}{\PYZsq{}}\PY{l+s+s1}{Id}\PY{l+s+s1}{\PYZsq{}}\PY{p}{]}
        
        \PY{c+c1}{\PYZsh{} Creamos un conjunto data que tiene ambos conjuntos, para así realizar las transformaciones a ambos también.}
        \PY{n}{data} \PY{o}{=} \PY{n}{pd}\PY{o}{.}\PY{n}{concat}\PY{p}{(}\PY{p}{[}\PY{n}{training}\PY{o}{.}\PY{n}{drop}\PY{p}{(}\PY{l+s+s1}{\PYZsq{}}\PY{l+s+s1}{SalePrice}\PY{l+s+s1}{\PYZsq{}}\PY{p}{,} \PY{n}{axis}\PY{o}{=}\PY{l+m+mi}{1}\PY{p}{)}\PY{p}{,} \PY{n}{testing}\PY{p}{]}\PY{p}{,} \PY{n}{keys}\PY{o}{=}\PY{p}{[}\PY{l+s+s1}{\PYZsq{}}\PY{l+s+s1}{train}\PY{l+s+s1}{\PYZsq{}}\PY{p}{,} \PY{l+s+s1}{\PYZsq{}}\PY{l+s+s1}{test}\PY{l+s+s1}{\PYZsq{}}\PY{p}{]}\PY{p}{)}
        
        \PY{c+c1}{\PYZsh{} Eliminamos el índice.}
        \PY{n}{data}\PY{o}{.}\PY{n}{drop}\PY{p}{(}\PY{p}{[}\PY{l+s+s1}{\PYZsq{}}\PY{l+s+s1}{Id}\PY{l+s+s1}{\PYZsq{}}\PY{p}{]}\PY{p}{,} \PY{n}{axis}\PY{o}{=}\PY{l+m+mi}{1}\PY{p}{,} \PY{n}{inplace}\PY{o}{=}\PY{k+kc}{True}\PY{p}{)}
\end{Verbatim}


    \subsubsection{3.3. Visualización de los datos
}\label{visualizaciuxf3n-de-los-datos}

Para visualizar los datos podemos utilizar varias de las funciones de
Pandas, como lo son:

\begin{itemize}
\tightlist
\item
  \texttt{pd.dataframe.head()}
\item
  \texttt{pd.dataframe.sample(5)}
\item
  \texttt{pd.dataframe.keys()}
\end{itemize}

    \begin{Verbatim}[commandchars=\\\{\}]
{\color{incolor}In [{\color{incolor} }]:} \PY{c+c1}{\PYZsh{} Ver los primeros 5 datos.}
        \PY{n}{data}\PY{o}{.}\PY{n}{head}\PY{p}{(}\PY{p}{)}
\end{Verbatim}


    \begin{Verbatim}[commandchars=\\\{\}]
{\color{incolor}In [{\color{incolor} }]:} \PY{c+c1}{\PYZsh{} Ver una muestra aleatoria de 5.}
        \PY{n}{data}\PY{o}{.}\PY{n}{sample}\PY{p}{(}\PY{l+m+mi}{5}\PY{p}{)}
\end{Verbatim}


    \begin{Verbatim}[commandchars=\\\{\}]
{\color{incolor}In [{\color{incolor} }]:} \PY{c+c1}{\PYZsh{} Ver los atributos.}
        \PY{n}{data}\PY{o}{.}\PY{n}{keys}\PY{p}{(}\PY{p}{)}
\end{Verbatim}


    Podemos ver información sobre el conjunto utilizando
\texttt{pd.dataframe.info()}

    \begin{Verbatim}[commandchars=\\\{\}]
{\color{incolor}In [{\color{incolor} }]:} \PY{c+c1}{\PYZsh{} Todos los datos.}
        \PY{n}{data}\PY{o}{.}\PY{n}{info}\PY{p}{(}\PY{p}{)}
\end{Verbatim}


    \begin{Verbatim}[commandchars=\\\{\}]
{\color{incolor}In [{\color{incolor} }]:} \PY{c+c1}{\PYZsh{} Conjunto de testing.}
        \PY{n}{testing}\PY{o}{.}\PY{n}{info}\PY{p}{(}\PY{p}{)}
\end{Verbatim}


    Esta función nos muestra los tipos de datos (enteros, reales, nominales,
etc.) y también la cantidad de elementos que tiene el conjunto, así como
también cuantos elementos tiene cada atributo.

Como resultado podemos ver que se tienen:

\begin{itemize}
\tightlist
\item
  Enteros \(\rightarrow\) 35
\item
  Reales \(\rightarrow\) 3
\item
  Nominales \(\rightarrow\) 43
\end{itemize}

    Podemos visualizar las estadísticas de los datos con la funcionalidad
\texttt{pd.dataframe.describe()} de Pandas.

    \begin{Verbatim}[commandchars=\\\{\}]
{\color{incolor}In [{\color{incolor} }]:} \PY{c+c1}{\PYZsh{} Ver estadísticas simples de los datos.}
        \PY{n}{data}\PY{o}{.}\PY{n}{describe}\PY{p}{(}\PY{p}{)}
\end{Verbatim}


    \paragraph{3.3.1. Pandas-profiling }\label{pandas-profiling}

    \begin{Verbatim}[commandchars=\\\{\}]
{\color{incolor}In [{\color{incolor} }]:} \PY{c+c1}{\PYZsh{} Generar resumen y mostrar.}
        \PY{n}{pandas\PYZus{}profiling}\PY{o}{.}\PY{n}{ProfileReport}\PY{p}{(}\PY{n}{training}\PY{p}{)}
\end{Verbatim}


    \subsubsection{3.4. Tratamiento de los datos
}\label{tratamiento-de-los-datos}

En este punto se realizan las transformaciones de los datos que se
adecuan a los modelos a utilizar.

    \paragraph{3.4.1. Sanitizar los datos }\label{sanitizar-los-datos}

La sanitización de los datos se utiliza ya que pueden haber instancias
que estén mal, en relación al tipo de dato. Este análisis es más a nivel
de negocio que de dato, ya que el negocio implica reglas que los datos
deben cumplir.

    \paragraph{3.4.2. Tratamiento datos faltantes
}\label{tratamiento-datos-faltantes}

Los datos faltantes son inadmitibles para muchos modelos. El tratamiento
de los datos faltantes implica imputar un valor, eliminar dichos datos o
eliminar el atributo.

    \paragraph{3.4.3. Tratamiento de outliers
}\label{tratamiento-de-outliers}

Hay muchos modelos en los cuales los outliers reducen la performance o
inducen un sesgo indeseado. Por eso, se debe detectar y tratar los
outliers para evitar este tipo de problemas.

    \paragraph{3.4.4. Correlación de atributos
}\label{correlaciuxf3n-de-atributos}

Hay muchos (como los modelos lineales) que la correlación de los
atributos influye fuertemente en los modelos. Es por esto, que muchas
veces se debe chequear la correlación de los atributos para así eliminar
los que están altamente correlacionados.

    \paragraph{3.4.5. Feature extraction }\label{feature-extraction}

Muchas veces se puede "diseñar" un atributo que es combinación de otros
atributos (lineal o no lineal) para así obtener más información. Tal
vez, este atrubto generado es un mejor predictor y se mejora la
solución.

    \paragraph{3.4.6. Transofrmaciones de los datos
}\label{transofrmaciones-de-los-datos}

En este punto se realizan las transofrmaciones necesarias de los datos.
Las transformaciones incluyen desde transformaciones para reducir el
sesgo o ajustar distribuciones de los datos hasta transformar los datos
a valores numericos o a valores categóricos. También en este punto se
incluye la estandarización y normalización si se debe hacer.

    \paragraph{3.4.7. Dimension reduction }\label{dimension-reduction}

Si las dimensiones son muy grantes, podemos aplicar técnicas que
reduzcan la dimensionalidad de los atributos.

    \subsection{\#\# 4. Modelado }\label{modelado}

En esta sección probamos varios modelos de machine learning. Utilizamos
la librería \textbf{sklearn}.

Utilizaremos los siguientes modelos: -
\texttt{sklearn.linear\_model.LinearRegression} \(\Rightarrow\)
http://scikit-learn.org/stable/modules/generated/sklearn.linear\_model.LinearRegression.html
- \texttt{sklearn.linear\_model.LassoCV} \(\Rightarrow\)
http://scikit-learn.org/stable/modules/generated/sklearn.linear\_model.LassoCV.html

    \begin{Verbatim}[commandchars=\\\{\}]
{\color{incolor}In [{\color{incolor} }]:} \PY{c+c1}{\PYZsh{} Modelos a utilizar:}
        \PY{k+kn}{from} \PY{n+nn}{sklearn}\PY{n+nn}{.}\PY{n+nn}{linear\PYZus{}model} \PY{k}{import} \PY{n}{LinearRegression}
        \PY{k+kn}{from} \PY{n+nn}{sklearn}\PY{n+nn}{.}\PY{n+nn}{linear\PYZus{}model} \PY{k}{import} \PY{n}{LassoCV}
\end{Verbatim}


    Y para chequear la performance de nuestros modelos utilizamos: *
\texttt{sklearn.metrics.make\_scorer} \(\Rightarrow\)
http://scikit-learn.org/stable/modules/generated/sklearn.metrics.make\_scorer.html
* \texttt{sklearn.metrics.accuracy\_score} \(\Rightarrow\)
http://scikit-learn.org/stable/modules/generated/sklearn.metrics.accuracy\_score.html
* \texttt{sklearn.model\_selection.cross\_val\_score} \(\Rightarrow\)
http://scikit-learn.org/stable/modules/generated/sklearn.model\_selection.cross\_val\_score.html

    \begin{Verbatim}[commandchars=\\\{\}]
{\color{incolor}In [{\color{incolor} }]:} \PY{k+kn}{from} \PY{n+nn}{sklearn}\PY{n+nn}{.}\PY{n+nn}{metrics} \PY{k}{import} \PY{n}{make\PYZus{}scorer}\PY{p}{,} \PY{n}{accuracy\PYZus{}score}
        \PY{k+kn}{from} \PY{n+nn}{sklearn}\PY{n+nn}{.}\PY{n+nn}{model\PYZus{}selection} \PY{k}{import} \PY{n}{cross\PYZus{}val\PYZus{}score}
\end{Verbatim}


    Utilizaremos \texttt{sklearn.model\_selection.GridSearchCV}
(http://scikit-learn.org/stable/modules/generated/sklearn.model\_selection.GridSearchCV.html)
para la optimización del modelo basado en fuerza bruta o
\texttt{sklearn-deap.evolutionary\_search.EvolutionaryAlgorithmSearchCV}
(https://github.com/rsteca/sklearn-deap) para la optimización del modelo
basado en algoritmos evolutivos.

    \begin{Verbatim}[commandchars=\\\{\}]
{\color{incolor}In [{\color{incolor} }]:} \PY{k+kn}{from} \PY{n+nn}{sklearn}\PY{n+nn}{.}\PY{n+nn}{model\PYZus{}selection} \PY{k}{import} \PY{n}{GridSearchCV}
        \PY{k+kn}{from} \PY{n+nn}{evolutionary\PYZus{}search} \PY{k}{import} \PY{n}{EvolutionaryAlgorithmSearchCV}
\end{Verbatim}


    Utilizaremos \texttt{sklearn.feature\_selection.SelectFromModel}
(http://scikit-learn.org/stable/modules/generated/sklearn.feature\_selection.SelectFromModel.html\#sklearn.feature\_selection.SelectFromModel)
para la optimización de los parámetros.

    \begin{Verbatim}[commandchars=\\\{\}]
{\color{incolor}In [{\color{incolor} }]:} \PY{k+kn}{from} \PY{n+nn}{sklearn}\PY{n+nn}{.}\PY{n+nn}{feature\PYZus{}selection} \PY{k}{import} \PY{n}{SelectFromModel}
\end{Verbatim}


    \subsubsection{4.1. Preparación del modelado
}\label{preparaciuxf3n-del-modelado}

Los conjuntos a utilizar en el modelado son:

    \begin{Verbatim}[commandchars=\\\{\}]
{\color{incolor}In [{\color{incolor} }]:} \PY{n+nb}{print}\PY{p}{(}\PY{l+s+s1}{\PYZsq{}}\PY{l+s+s1}{X\PYZus{}test:}\PY{l+s+s1}{\PYZsq{}}\PY{p}{,} \PY{n+nb}{len}\PY{p}{(}\PY{n}{X\PYZus{}test}\PY{p}{)}\PY{p}{)}
        \PY{n+nb}{print}\PY{p}{(}\PY{l+s+s1}{\PYZsq{}}\PY{l+s+s1}{X\PYZus{}train:}\PY{l+s+s1}{\PYZsq{}}\PY{p}{,} \PY{n+nb}{len}\PY{p}{(}\PY{n}{X\PYZus{}train}\PY{p}{)}\PY{p}{)}
        \PY{n+nb}{print}\PY{p}{(}\PY{l+s+s1}{\PYZsq{}}\PY{l+s+s1}{y\PYZus{}train:}\PY{l+s+s1}{\PYZsq{}}\PY{p}{,} \PY{n+nb}{len}\PY{p}{(}\PY{n}{y\PYZus{}train}\PY{p}{)}\PY{p}{)}
\end{Verbatim}


    \subsubsection{4.2. Entrenamiento de los modelos
}\label{entrenamiento-de-los-modelos}

En esta sección se entrenan los modelos especificados anteriormente para
ver cuál es el que realiza la mejor predicción.

    \paragraph{4.2.1. Linear Regression }\label{linear-regression}

    \begin{Verbatim}[commandchars=\\\{\}]
{\color{incolor}In [{\color{incolor} }]:} \PY{c+c1}{\PYZsh{} Cargamos el modelo.}
        \PY{n}{linreg\PYZus{}clf} \PY{o}{=} \PY{n}{LinearRegression}\PY{p}{(}\PY{p}{)}
        
        \PY{c+c1}{\PYZsh{} Entrenamos el modelo.}
        \PY{n}{linreg\PYZus{}clf}\PY{o}{.}\PY{n}{fit}\PY{p}{(}\PY{n}{X\PYZus{}train}\PY{p}{,} \PY{n}{y\PYZus{}train}\PY{p}{)}
        
        \PY{c+c1}{\PYZsh{} Validación cruzada.}
        \PY{n}{acc\PYZus{}linreg} \PY{o}{=} \PY{n}{cross\PYZus{}val\PYZus{}score}\PY{p}{(}\PY{n}{linreg\PYZus{}clf}\PY{p}{,} \PY{n}{X\PYZus{}train}\PY{p}{,} \PY{n}{y\PYZus{}train}\PY{p}{,} \PY{n}{cv}\PY{o}{=}\PY{l+m+mi}{10}\PY{p}{)}\PY{o}{.}\PY{n}{mean}\PY{p}{(}\PY{p}{)}
        \PY{n+nb}{print}\PY{p}{(}\PY{n}{acc\PYZus{}linreg}\PY{p}{)}
\end{Verbatim}


    \paragraph{4.2.2. Linear Regression, L1 Regularisation
}\label{linear-regression-l1-regularisation}

    \begin{Verbatim}[commandchars=\\\{\}]
{\color{incolor}In [{\color{incolor} }]:} \PY{c+c1}{\PYZsh{} Cargamos el modelo.}
        \PY{n}{linl1\PYZus{}clf} \PY{o}{=} \PY{n}{LassoCV}\PY{p}{(}\PY{p}{)}
        
        \PY{c+c1}{\PYZsh{} Entrenamos el modelo.}
        \PY{n}{linl1\PYZus{}clf}\PY{o}{.}\PY{n}{fit}\PY{p}{(}\PY{n}{X\PYZus{}train}\PY{p}{,} \PY{n}{y\PYZus{}train}\PY{p}{)}
        
        \PY{c+c1}{\PYZsh{} Validación cruzada.}
        \PY{n}{acc\PYZus{}linl1} \PY{o}{=} \PY{n}{cross\PYZus{}val\PYZus{}score}\PY{p}{(}\PY{n}{linl1\PYZus{}clf}\PY{p}{,} \PY{n}{X\PYZus{}train}\PY{p}{,} \PY{n}{y\PYZus{}train}\PY{p}{,} \PY{n}{cv}\PY{o}{=}\PY{l+m+mi}{10}\PY{p}{)}\PY{o}{.}\PY{n}{mean}\PY{p}{(}\PY{p}{)}
        \PY{n+nb}{print}\PY{p}{(}\PY{n}{acc\PYZus{}linl1}\PY{p}{)}
\end{Verbatim}


    \subsubsection{4.3. Comparación de modelos
}\label{comparaciuxf3n-de-modelos}

Una vez que tenemos las precisiones de los modelos, podemos comparar las
performance de los modelos.

    \begin{Verbatim}[commandchars=\\\{\}]
{\color{incolor}In [{\color{incolor} }]:} \PY{c+c1}{\PYZsh{} Creamos un dataframe para mostrar los datos.}
        \PY{n}{model\PYZus{}performance} \PY{o}{=} \PY{n}{pd}\PY{o}{.}\PY{n}{DataFrame}\PY{p}{(}\PY{p}{\PYZob{}}
            \PY{l+s+s2}{\PYZdq{}}\PY{l+s+s2}{Model}\PY{l+s+s2}{\PYZdq{}}\PY{p}{:} \PY{p}{[}\PY{l+s+s2}{\PYZdq{}}\PY{l+s+s2}{Linear Regression}\PY{l+s+s2}{\PYZdq{}}\PY{p}{,} \PY{l+s+s2}{\PYZdq{}}\PY{l+s+s2}{Linear Regression L1}\PY{l+s+s2}{\PYZdq{}}\PY{p}{]}\PY{p}{,}
            \PY{l+s+s2}{\PYZdq{}}\PY{l+s+s2}{Accuracy}\PY{l+s+s2}{\PYZdq{}}\PY{p}{:} \PY{p}{[}\PY{n}{acc\PYZus{}linreg}\PY{p}{,} \PY{n}{acc\PYZus{}linl1}\PY{p}{]}
        \PY{p}{\PYZcb{}}\PY{p}{)}
        
        \PY{c+c1}{\PYZsh{} Ordenamos de mayor a menor.}
        \PY{n}{model\PYZus{}performance}\PY{o}{.}\PY{n}{sort\PYZus{}values}\PY{p}{(}\PY{n}{by}\PY{o}{=}\PY{l+s+s2}{\PYZdq{}}\PY{l+s+s2}{Accuracy}\PY{l+s+s2}{\PYZdq{}}\PY{p}{,} \PY{n}{ascending}\PY{o}{=}\PY{k+kc}{False}\PY{p}{)}
\end{Verbatim}


    \subsubsection{4.4. Feature selection }\label{feature-selection}

Una vez que tenemos el modelo que nos de la mejor predición, realizamos
la selección de atributos utilizando algoritmos evolutivos para observar
si se mejora en la predicción o no. En si, esto se debería realizar para
cada algoritmo, y luego compararlos, pero consume mucho tiempo este tipo
de procesos.

    \subsubsection{4.5. Optimización }\label{optimizaciuxf3n}

En esta sección se optimiza el modelo para utilizarlo en la puesta a
producción.

    \subsection{\#\# 5. Evaluación }\label{evaluaciuxf3n}

    \subsection{\#\# 6. Puesta en producción
}\label{puesta-en-producciuxf3n}


    % Add a bibliography block to the postdoc
    
    
    
    \end{document}
